PUXELS \cite{kauker2013rendering}
\begin{itemize}
  \item Performance drop of SES rendering on newer hardware (GF 680 GTX) --- we can perform better. We contacted authors, they do not know exactly why, they think it is change in the internal architecture between Fermi and Kepler (AJ)
  \item Slow rendering of SAS. Too many layers in pixels --- we can do better by surface layers detection (AJ)
\end{itemize}

There are more solutions taken from computer graphics; e.g., OIT (JP)

AOOM
\begin{itemize}
  \item We employ transparency modulation techniques presented in their paper \cite{borland2011ambient}
\end{itemize}

\subsection{Molecular Surface Representation}
There are several types of molecular surface representation proposed in the literature~\cite{STAR2015}. Nevertheless, for cavity analysis,  the solvent-excluded surface (SES) belongs to the most used representation~\cite{todo} amongst biologists. This representation allows us to directly asses whether a solvent, approximated a sphere of a certain radius, is able to reach a binding site of interest on the molecular surface. Such a binding site is located inside a cavity or a tunnel, while the molecular surface might contain tens of cavities per a single simulation snap-shot. Additionally, computation of SES is not a trivial task, which also requires a substantial computation and algorithmic capabilities. Therefore, it would be essential to posses a technique that could provide us an instant computation, and an interactive and meaningful visualization of the cavities in the context of molecular surface.
\subsection{Extraction and Visualization of Cavities and Tunnels}

